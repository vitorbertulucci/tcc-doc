\begin{resumo}
  A digitalização de processos judiciais brasileiros é uma tarefa que vem sendo realizada desde a oficialização das plataformas virtuais para controle do fluxo processual, como o PJe. Porém, o alto volume de processos digitalizados e os diversos tipos de ferramentas utilizadas no processo de digitalização podem formar imagens com ruídos, muitas vezes impossibilitando a leitura e entendimento do conteúdo presente, o que dificulta na extração do conteúdo das mesmas por meio de OCR. Visto isso, é proposto a construção de algoritmos de \textit{Deep Learning} utilizando as arquiteturas de CycleGAN e \textit{decrappification} para servirem como pré-processadores de imagens escaneadas com o intuito de impactar positivamente na qualidade dos textos extraídos via algoritmos de OCR frente aos processos do Supremo Tribunal Federal. Tal base de dados consiste 354.501 páginas de documentos processuais, sendo 80.624 delas imagens digitalizadas. Os algoritmos de CycleGAN e \textit{decrappification} foram testados em 1000 imagens ruins e atingiram X\% e Y\% de melhora no OCR frente a Z\% em algoritmos convencionais também testados.

%  O resumo deve ressaltar o objetivo, o método, os resultados e as conclusões
%  do documento. A ordem e a extensão
%  destes itens dependem do tipo de resumo (informativo ou indicativo) e do
%  tratamento que cada item recebe no documento original. O resumo deve ser
%  precedido da referência do documento, com exceção do resumo inserido no
%  próprio documento. (\ldots) As palavras-chave devem figurar logo abaixo do
%  resumo, antecedidas da expressão Palavras-chave:, separadas entre si por
%  ponto e finalizadas também por ponto. O texto pode conter no mínimo 150 e
%  no máximo 500 palavras, é aconselhável que sejam utilizadas 200 palavras.
%  E não se separa o texto do resumo em parágrafos.

 \vspace{\onelineskip}

 \noindent
 \textbf{Palavras-chaves}: pré-processador. OCR. redes neurais.
\end{resumo}
