\chapter[Introdução]{Introdução}
% \addcontentsline{toc}{chapter}{Introdução}

A informatização dos processos judiciais é um processo de automação que vem sendo utilizado, atualizado e evoluido desde os anos 90 nos Estados Unidos \cite{automating-judicial-doc} e, no Brasil, a partir de 2006 com a publicação da Lei nº 11.419, sancionada pelo então presidente Luiz Inácio Lula da Silva, permitindo a tramitação de processos judiciais em ambientes eletrônicos \cite{digitalizacao-de-proc-judiciais}. Tal lei possibilitou informatizar qualquer forma de armazenamento e tráfego de informações referentes aos processos em arquivos digitais.

Em 2010, o Superior Tribunal de Justiça tornou-se o primeiro tribunal brasileiro totalmente virtualizado, com todos os seus processos disponíveis em plataformas online \cite{digitalizacao-de-proc-judiciais}, onde a maior parte desses processos foi transferida para a plataforma lançada oficialmente em 2011 pelo Conselho Nacional de Justiça (CNJ), o PJe ou Processo Judicial Eletrônico. Tal plataforma permitiu não só a facilidade no trabalho dos advogados, juízes, procuradores e estagiários com a sua mudança de paradigma de trabalho como também auxiliou na diminuição do tempo de vida de boa parte dos processos \cite{pje-diminuicao-tempo-do=proc}, onde, segundo o CNJ em seu relatório \textit{Justiça em Números} de 2018 sobre os dados de 2017, mostram que o tempo de vida médio dos processos podem chegar a 7 anos e 11 meses na fase de execução da Justiça Federal.

Contudo, a adoção de um modelo mais inovador de trabalho para o ramo jurídico não soluciona o problema de criação de novos processos físicos e nem dos processos físicos ainda em vigência e, por isso, é necessário uma etapa de digitalização dos mesmos. Tal atividade, manual e exaustiva, é preciso passar os documentos por \textit{scanners} para permitir o uso dos mesmos em sistemas digitais, como o PJe, sem contar que com isso, o processo passa a ser digitalizado e não de fato digital, ou seja, não é possível selecionar o texto de um documento digitalizado, extrair informações, etc. Documentos digitalizados nada mais são do que fotos em páginas de PDF, limitando assim o uso das informações das mesmas de maneira simplificada.

Para solucionar uma parte dos problemas da digitalização e extração de texto a partir de imagens, utiliza-se uma tecnologia denominada \textit{Optical Character Recognition} ou OCR onde, a partir dos documentos digitalizados, os mesmos são colocados em um processamento utilizando ORC para extrair texto a partir das imagens dos documentos digitais. Porém, tal tecnologia precisa de uma imagem limpa e clara do conteúdo nela presente, o que nem sempre é verdade.

\newpage

\section{Objetivo}

Tendo em vista a problemática apresentada a respeito do processo de informatização dos documentos jurídicos, o objetivo principal desse trabalho de conclusão de curso é apresentar um novo modelo que pode vir a auxiliar as tecnologias de OCR a apresentarem melhores resultados, fazendo o uso de 


% Este documento apresenta considerações gerais e preliminares relacionadas 
% à redação de relatórios de Projeto de Graduação da Faculdade UnB Gama 
% (FGA). São abordados os diferentes aspectos sobre a estrutura do trabalho, 
% uso de programas de auxilio a edição, tiragem de cópias, encadernação, etc.

% Este template é uma adaptação do ABNTeX2\footnote{\url{https://github.com/abntex/abntex2/}}.
